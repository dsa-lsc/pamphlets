\documentclass[10pt]{memoir}
\usepackage[
paperwidth=5.5in,
paperheight=8in,
inner=0.5in,
outer=0.8in,
top=0.8in,
bottom=0.7in
]{geometry}
%\usepackage[print,1to1]{booklet}


\usepackage{microtype}
\usepackage[T1]{fontenc}
\renewcommand{\familydefault}{\sfdefault}

\setlength{\parindent}{0pt}
\setlength{\parskip}{12pt}
\pagestyle{myheadings}
%\usepackage{hyperref}

\def \href #1#2{#2} %ignore \href

\begin{document}


\subsection{Preamble}\label{preamble}

\begin{quote}
``We believe in the socialist principles of common ownership and that
worker control over workplaces can only be advanced through the creation
and support of worker-owned firms, radical trade unions, workers' and
neighborhood councils, popular assemblies, credit unions and alternative
banking systems, community land trusts, and other directly democratic
non-state institutions. The power of socialist parties and socialist
governments should be subordinated to these more decentralized
grassroots formations.''

From \href{https://dsa-lsc.org/}{the founding statement} of the
Libertarian Socialist\\ Caucus, 2017
\end{quote}

The Libertarian Socialist Caucus of the Democratic Socialists of America
aspires to create a socialist society freed from all forms of hierarchy
and domination. The question has been raised from both inside and
outside of the Caucus: how do we get from here to there?

Since the Caucus's founding, we have been engaged in the long process of
exploring and experimenting to find an answer to this question. Though
we came from all walks of life and from across the country, our faith in
radical democracy meant that we were confident a shared practice and a
common struggle would allow us to slowly find our way together. After
more than a year of discussing our experiences and ideas, organizing
projects within and between our locals, and building lasting
institutions both inside and outside of the DSA, we have finally arrived
at a consensus as to the broad outlines of a revolutionary strategy that
fits our current context and material conditions. It is now our shared
view that the path to socialism in our time is to build \textbf{dual
power}.

\subsection{The Current Situation}\label{the-current-situation}

Since Occupy Wall Street, socialist organizing has been on the rise in
the United States. In its wake, a series of increasingly radical
movements demanding the transformation of society from the ground up
have emerged. With the 2016 election, this slow but steady movement
ramped up so that we now have the most active and vibrant socialist
movement in America since the 1960's. As we've grown more numerous and
more powerful, the question of strategy looms large. What kinds of
victories are actually possible under our present circumstances? Given
the balance of power between us and capital and its lackeys, what
concrete actions are to be done so that socialism can be won? What do
age-old debates about reform vs.~revolution have to tell us---and can
they help us at all? These are the critical questions comrades are
asking across our movement, and much depends on the answers.

The path we have chosen to undertake begins with the realization that
the State has failed us.

There was a time when it seemed to many that socialism could tame the
State and use it to create a ``welfare state'' for the benefit of all.
Yet before most of us were even born, the State was fully recaptured by
capital, which only offers us crumbs and the rusted scrapings of
neoliberalism, the reigning capitalist mode of governance. Neoliberalism
is the latest scheme that the bosses have constructed to steal what
rightfully belongs to us all. It is the capitalist reaction to the
preceding social-democratic order that was forged in the wake of the
Great Depression and the cataclysm of world war. At that time, these
disasters finally pushed the capitalist class to compromise with the
working classes, creating a system which allowed some (but hardly all)
ordinary people a decent standard of living for the first time under
capitalism.

Neoliberalism was the brutal and ruthless counterrevolution that was
birthed in the 1970's against these lukewarm reforms. From public goods
seized by the capitalist class in the name of ``privatization'' to the
defanging and destruction of industrial democracy, from setting the
workers of the world against each other in bids for the lowest wages to
the brutality of the speed-up and hyper-surveillance which are now
ubiquitous in the workplace, neoliberal capitalism has destroyed any
possibility of a decent life for the common person. The capitalist
crisis is the gnawing, depressing constant of our daily lives that we
have sadly grown accustomed to. What's more, the all-consuming
grow-or-die imperative of capitalism is unsustainable, and neither our
species nor the planet can survive if things continue as they are.
Unless production is socialized, with vast amounts of industry becoming
automated, this technological process will only serve to benefit the
elites. We need strategies that work to effectively construct the
liberated world we wish to see, a world which prioritizes all biological
life over bloated, opaque, private power and the reckless profit seeking
of the ruling class. We believe that the State cannot build that world
for us, that we must build it for ourselves.

\subsection{Our Vision of Socialism}\label{our-vision-of-socialism}

It helps first to know what we're fighting for. So let's define our
vision of the socialist society we want to build.

We seek to abolish capitalist commodity production and wage labor---that
is, the system in which ordinary people are deprived of their means of
survival and forced to work under the dictatorial command of the rich in
exchange for tokens (usually too few) that we must use to purchase our
basic needs---and replace these with a system where the key
infrastructure of society is owned in common, managed through direct
democracy by the people themselves, and used to produce and allocate
goods and services to each equitably and according to their needs. In
broad terms, the power to make decisions in a socialist society should
reside with those affected by them: workers should run their workplaces
and communities their common affairs, with production and investment
directed by and for all those involved.

Even further than this, we aim to construct a socialism which attacks
all hierarchies and forms of domination. Our socialism is not just the
socialism of industrial factory workers, but of all workers---including
those who produce culture, those whose care work cultivates the human
beings who reproduce our society, and those who cannot work at all. We
must abolish prisons and the carceral state and replace these with
restorative justice, mediation, de-escalation, rehabilitation, and
conflict resolution. We must end imperialism and abolish the militarist
state, replacing these with a system of international cooperation
between equals that can heal the divides between the many communities of
the planet and raise living standards in places underdeveloped and
overextracted by colonialist and capitalist exploitation. We need to
burn down patriarchy and abolish the racial caste system, replacing
these with gender parity, racial equity, and democratic pluralism at all
levels of leadership and decision making. Finally, we must work to
pursue a rapid, just transition away from a fundamentally unsustainable
fossil fuel capitalism whose hunger for profits is destroying our shared
ecology---not only through climate change but also mass extinction of
nonhuman species, ocean acidification, disastrous levels of pollution,
and more---to move us toward an ecosocialism capable of rebalancing the
needs of all humans with our obligations to the nonhuman world from
which we sprang.

That's the world we want to build. Now to return to the question of how
we'll build it.

To accomplish these things as a movement of the working classes in all
our variety, we must organize with all who are exploited and oppressed
by the capitalist system. That means working together not just in the
workplace, but in our communities (online and in real life), our blocks
and our prisons, our schools and our neighborhoods, our homes and our
streets, to build grassroots working-class power. We recognize that this
includes those workers engaged formally and informally at the point of
production, logistics, and realization, but also those who are
unemployed, retired, incarcerated, dependent, or disabled, and all those
who do not own and control the means of capitalist production as part of
the 1\% or their lackeys.

Ours is an emergent strategy that will unfold in unique ways in a
variety of different contexts. The struggle will be different in
different places, and our tactics will have to change accordingly.
Nevertheless, we believe that a shared path has opened up in struggles
around the world, and this is the one we wish to pursue here in the
United States. It consists of building our way toward our ultimate goal
of libertarian socialism, assembling it piece by piece. We believe our
current projects and pursuits must mirror---and, in mirroring,
become---the world we want to emerge from the ashes of capitalism. In
short, our method consists of embodying the world we dare to dream.

\subsection{What Is Dual Power --- And How Do We Build
It?}\label{what-is-dual-power-and-how-do-we-build-it}

\begin{quote}
``Self-management will only be possible if people's attitudes to social
organization alter radically. This in turn, will only take place if
social institutions become a meaningful part of their real daily life.''

\href{https://www.marxists.org/archive/castoriadis/1972/workers-councils.htm}{``Workers'
Councils and the Economics of a Self-Managed Society''} by Cornelius
Castoriadis, 1972
\end{quote}

Let's get specific.

How do we effectively build political space where direct democracy,
mutual aid, solidarity, and an ecologically sustainable human existence
can prevail? To start with, we need to be able to provide for our
immediate needs. In doing so, we must organize to seize control of
powerful nodes of production, reproduction, and realization while
simultaneously cultivating models of the society we wish to live in.

\textbf{Dual power} is a strategy that builds liberated spaces and
creates institutions grounded in direct democracy. Together these spaces
and institutions expand into the ever widening formation of a new world
``in the shell of the old.'' As the movement grows more powerful, it can
engage in ever larger confrontations with the ruling class---and
ultimately a contest for legitimacy against the institutions of
capitalist society.

In our view, dual power is comprised of two component parts: (1.)
\textbf{building counter-institutions} that serve as alternatives to the
institutions currently governing production, investment, and social life
under capitalism, and (2.) organizing through and \textbf{confederating
these institutions} to \textbf{build up a base of grassroots
counter-power} which can eventually challenge the existing power of
capitalists and the State head-on. In the short term, such a strategy
helps win victories that improve working people's standard of living,
helps us meet our needs that are currently left unaddressed under
capitalism, and gives us more of a say over our day-to-day lives. But
more excitingly, in the long run these methods provide models for new
ways of organizing our society based on libertarian socialist
principles. They create a path toward a revolutionary transition from a
capitalist mode of production. This revolution will liberate us from
both the need and the drive to create wealth for the rich, making
possible a socialist mode of production that seeks to benefit all of
humanity and free us from the lonely confines of commodity
relationships.

The Libertarian Socialist Caucus is organizing to build networks of
community councils, popular assemblies, tenant unions, and other bodies
of participatory democracy that form a counterweight to the
authoritarian institutions presently governing our lives, organizing
society in parallel against capitalism and the State. Democratic labor
unions can seize the workplace; worker-owned cooperatives can build it
anew in democratic form; tenant unions can take control of housing; our
councils and assemblies can restructure political authority around our
own processes of confederal direct democracy. \_This framework of
building popular power outside the governing institutions of our present
system, to challenge and eventually displace those institutions with
truly democratic ones of our own making, is the heart of \textbf{dual
power}.

\subsection{Counter-Institutions And
Counter-Power}\label{counter-institutions-and-counter-power}

There are many examples of various counter-institutions, but they all
share some core characteristics: they are directly democratic, are
created and run by the people who benefit from them, and are independent
of control by the State and capital alike. By building these
organizations, working-class people can create a new form of social,
political, and economic power that exists in tension and opposition to
the power of capitalism and the State. Counter-institutions can include,
but are not limited to: community councils, popular neighborhood
assemblies, worker's councils, syndicalist unions, rank-and-file trade
unions, worker-owned cooperatives, locally and regionally networked
redistributive solidarity economies, participatory budgeting
initiatives, and time banks. They also include collectives committed to
the provision of mutual aid and disaster relief, tenant unions,
community land trusts, cooperative housing, communal agriculture and
food distribution systems, community-owned energy, horizontal education
models, childcare collectives, and community-run health clinics, to name
a few. These structures cannot exist in isolation but must actively
network and support one another across communities and regions. Where
this dynamic is newly emerging, counter-institutions must strive to
support the creation and fostering of similar organizing. When possible,
these counter-institutions link up politically, economically, and
socially to form a self-sufficient ecosystem; and ultimately,
confederate into direct-democratic political bodies \textbf{in and
across communities all over the world}.

Our goal in building up this infrastructure is to create
\textbf{counter-power}. Counter-power is our ability to
\textbf{delegitimize, disrupt,} and \textbf{demonstrate our power
against} the current regime by developing and deploying cutting-edge
cultural and organizational practices. These practices form part of the
direct action toolbox which can collectively be used to undermine and
delegitimize social, political, and economic hierarchies while
demonstrating working-class power and forging new narratives rooted in
solidarity. Working-class communities that organize to take care of
various issues affecting them---from street repair to food distribution,
brake-light clinics to lead clean-up in neglected buildings---all show
the limits of the neoliberal state's ability to solve our problems and
thus \textbf{delegitimize} it in the eyes of observers. Work strikes and
stoppages, rent strikes, highway blockades, and mass demonstrations that
overwhelm the authorities' ability to maintain ``business as usual'' are
all part of how the working class \textbf{demonstrates its power}.
Serious \textbf{disruption}, when it proves necessary, requires that we
first develop the capability to organize large actions such as general
strikes, factory or other infrastructure seizures, and mass uprisings
that establish autonomous areas of working-class organization and bases
for mobilizing to take control of our whole society.

Once these methods combine and embolden a large and organized mass base,
they represent a direct contestation of the ruling institutions of
society, and we have then entered into a situation of true \textbf{dual
power}. At this stage, the ``powers that be'' are rivaled by a
counter-power which has become strong enough to provide the real
possibility of overthrowing the existing social order, and it becomes
unthinkable to take even one step back.

We believe that countless alternatives are already sprouting up in the
cracks of the capitalist system, and that these must be nurtured in
order to blossom into a free, democratic, and just world. The old system
will not fall from any single blow; instead we must constantly be
probing, experimenting with new iterations of dual power organizational
forms, until we have created an irresistible set of concrete facts on
the ground whereby the new, liberated world competes for legitimacy
against the old, dying, and illegitimate order.

\subsection{Key Sites of Struggle}\label{key-sites-of-struggle}

What does a dual power strategy look like on a grassroots level? As
we've already discussed, the strategy is designed to meet particular
needs in specific contexts; this will necessarily vary from place to
place and movement to movement. In a sense, what's more important about
dual power is the general principle around which organizers can design
specific interventions in their own communities.

That said, it's helpful for the purposes of illustration to walk through
some concrete examples. To that end, here are a few key sites of
struggle that our Caucus has identified through our activity and ways we
think we can fight for libertarian socialist dual power.

\subsubsection{Industrial Democracy}\label{industrial-democracy}

In modern capitalism, the owners and bosses use their tools of state and
corporate coercion to force us to hand over our time. On threat of
starvation they force us to live and breathe on their clock for a shitty
bargain: a wage set by a our masters that is as low as our collective
power will allow, and which is closer and closer to the bare minimum
level for a person in any given community to be able to survive. These
wages are little more than a reminder of how much time we have lost by
making profits for them---profits which they can command without effort,
tossing the leftovers at those of us who have produced it (if only just
enough to get us to show up to work the next day). This is the world we
live in, where in the heart of ``democratic'' societies, most people
spend most of their days toiling in workplaces which are absolute
dictatorships. In these tyrannical workplaces where we have no say in
what gets done or how to do it, our natural creativity and initiative is
drowned in the despair of a world where the only real choice is between
wagelessness and a sophisticated form of slavery. Under neoliberalism,
our labor unions have been continually subjected to waves of legal
assaults that chip away at the marginal legal power that they built up
during the New Deal era. As the labor movement was stripped of power,
bureaucratized, and removed from the shop floor into the ``halls of
government,'' the unions became alienated from their original goal,
namely building working-class power and extending the ideal of democracy
into the workplace.

We can fight back in our communities and workplaces through organizing
with our fellow workers, listening to their concerns and building
collective power with immediate material demands as well as providing
our vision for the revolutionary overthrow of capital and all its
associated oppressions. As socialists and syndicalists we can salt into
non-union shops, dual card, embed ourselves into already existing
unions, and consistently call on the unions to build true industrial
democracy. We can boldly proclaim that we will not tolerate acquiescence
by union bureaucrats who, more often than not, answer to the capitalist
class through backroom bargaining while growing comfortable on large
salaries rather than defending the interests of the working-class rank
and file.

It's no longer enough for unions to demand higher wages and safer
working conditions: they must return to their roots and agitate for
direct worker control of production and investment. \textbf{Industrial
democratic unionism} builds toward radical self-management and direct
democracy of workers on the shop floor. We believe that the ultimate
goal of union agitation must be \textbf{direct democracy in the
workplace}.

There already exist models of workplace direct action and of building
workers councils \textbf{within and outside of} the current union
framework which point to places where workers can reassert control over
the means of production. For example, in workplaces where many unions
work side by side in a context with many unorganized workers, impromptu
committees of rank-and-file workers can coordinate solidarity and direct
action that union bureaucracy and labor law is designed to stifle. On
the other hand, large and established unions of state workers are often
the only organizations presently strong enough to exert serious
counter-power in fighting the intensified destruction of the commons.
\href{https://dsa-lsc.org/2018/03/17/statement-in-solidarity-with-striking-teachers/}{Wildcat
strikes by teachers across ``right to work'' states has finally
demonstrated} the power of workers over and above our state-sanctioned
rights. Rank-and-file dissident movements within highly bureaucratic and
oligarchical unions like the Teamsters have
\href{https://www.jacobinmag.com/2018/10/ups-contract-rejection-james-hoffa-hybrid-drivers}{fought}
to take over unions and democratize them so that they will be run by the
workers themselves and actually fight for their class interests.
Finally, syndicalist unions like the Industrial Workers of the World
(I.W.W.) are already run as confederated direct democracies and have
\href{https://www.oregonlive.com/expo/erry-2018/05/2d74cccd872709/burgerville_fastfood_workers_u.html}{won
crucial victories organizing new shops}, particularly in precarious,
low-wage industries like fast food which the established ``business
unions'' have largely avoided. In other countries, syndicalist unions
like Spain's \href{http://cgt.org.es/}{Confederación General del
Trabajo} organize
\href{http://progressivespain.com/2018/01/08/profile-confederacion-general-del-trabajo-cgt-general-confederation-of-labour/}{as
many as 80,000 members}, host their own
\href{https://www.youtube.com/channel/UCAdtgt5slNceh_0z2dQqpLw}{TV news
shows}, and confederate into internationals like the new
\href{https://freedomnews.org.uk/founding-of-a-new-international/}{International
Confederation of Labor}, showing that confederal direct democracy is
more than capable of reaching a scale rivaling that of the biggest
unions.

All of these options should be on the table. Libertarian socialists must
craft a new industrial democratic unionism using whatever tools and
tactics fit local contexts. The capitalists have used everything at
their disposal to ensure their hegemonic reign over the working
class---where our lives are dominated by our work---and in order to
create bottom-up socialism, we must organize in our workplaces. Our
tactics are direct action, strikes, sit downs, and slowdowns to demand
deliveries for immediate on-the-job gains, all to give the bosses an
offer they can't refuse. Those tactics in turn will only be possible if
we create democratic, militant and independent organizations of workers.
The infrastructure of industrial democracy can only be built by creating
\textbf{dual power}.

\subsubsection{Solidarity Economy}\label{solidarity-economy}

The goal of syndicalism is traditionally worker control over production,
and its classic model is for the workers to simply take over their
workplaces in a revolutionary situation. But short of that, what can
radical workers do in the meantime? And are there potentially piecemeal,
legal ways that workers can wrest control over their jobs?

Solidarity Economy initiatives are dedicated to seeding worker-owned
cooperatives, economic democracy projects, time banking, community land
trusts, publicly owned and democratically self-managed socialist
enterprises, and a variety of cooperative economic endeavors federated
throughout the country and internationally. Dual power organizing
includes creating and networking alternative libertarian socialist
enterprises that are rooted in principles of economic justice, worker
control, and internal democracy. Worker-owned cooperatives, committed to
increasing democracy within the economy begin in the workplace.
\textbf{Worker ownership and control is a principal objective of dual
power.}

Cooperatives by themselves however, are not enough. It is necessary to
bring a socialist vision to any economic enterprise, and that these
enterprises are intrinsically tied directly to our communities in need,
ensuring a larger vision of ecological and communal health. Here, too,
the work is already beginning to be done. Cooperativist movements have
sprung up in many of the most neglected communities in the United States
often led by working-class people of color seeking to revitalize
neighborhoods and cities left completely desolate by
deindustrialization, white flight, and systemic disinvestment. In this
context, cooperatives are a way not only of putting power in the hands
of workers, but of creating a new ecosystem of interdependent
enterprises and financial institutions, all of them under democratic
control. These endeavors can get a dead economy moving again, create
employment which transcends the wages system through worker-ownership,
build sustainable food and energy sovereignty, and lay the groundwork
for a just transition into ecological sustainability.
\href{https://www.theworkingworld.org/us/}{The Working World} and the
\href{https://bcdi.nyc/}{Bronx Cooperative Development Initiative} are
two examples of organizations that support this work on a mass scale. In
\href{https://cooperationjackson.org/}{Cooperation Jackson}, a radical
movement openly attempting to build dual power in Mississippi, such
economic development through creating a solidarity economy is
\href{https://static1.squarespace.com/static/59826532e6f2e1038a2870ff/t/5b6655c86d2a73ce52f8d6ce/1533433296830/Jackson-Kush+Plan.pdf}{explicitly
tied} to a revolutionary political program directed by ongoing people's
assemblies.

We also need to establish standards for how cooperatives are run
internally. In order for our vision to remain committed to building
these as socialist institutions, we must not emulate the traditional
capitalist firm, which is highly competitive and extracts wealth to the
top while reproducing the social, political, and economic hierarchy of
owner and bosses over laborers. We must remain committed to the
immediate benefit of our communities within a larger ecosystem of a
cooperative commonwealth. The basis of our larger socialist vision must
be firmly situated in community- and worker-management of our own
social, economic, and political institutions of direct democracy,
solidarity, and mutual aid. Therefore, in order to ensure the socialist
nature of a solidarity economy network, these cooperative enterprises
must be tied to direct-democratic community councils and assemblies and
play a redistributive or social role in any given community,
\textbf{laying the foundations for the revolutionary transition toward a
bottom-up socialism}.

\subsubsection{Community Councils and
Assemblies}\label{community-councils-and-assemblies}

The building of community councils or assemblies can take a multitude of
approaches but typically falls into two categories:

\begin{enumerate}
\def\labelenumi{\arabic{enumi}.}
\itemsep1pt\parskip0pt\parsep0pt
\item
  Entering into \textbf{already existing} civic institutions,
  neighborhood councils, associations, block clubs, etc, and injecting
  them with direct-democratic, anti-authoritarian, and anti-capitalist
  goals and processes.
\item
  \textbf{Creating our own councils} and popular assemblies based on
  material conditions that demand action. This option requires that we
  assess the needs and conditions of communities, canvass door to door,
  and meet people where they are at. We must first identify urgent
  problems in the community before calling for public assemblies focused
  on addressing these problems, building engaged communities to combat
  the alienation of capitalism.
\end{enumerate}

Some of these assemblies will evolve around single-issue campaigns that
dissolve when the goal is met; others will build into long-term,
permanent counter-power institutions. When people realize the collective
power that materializes when we work together, we can choose to embody
our ideals by making more and more decisions collectively in a
direct-democratic assembly. This would allow us to live our lives with
meaning and empowerment, leaving behind the paltry trappings of
consumerism, anti-democratic top-down workplaces, spiritual hollowness
and environmental degradation that currently afflict our hearts and
minds.

These popular councils and assemblies aren't just nice things to have;
they are the building blocks of a bottom-up socialism at the community
level. The best propaganda for libertarian socialism is not any
philosophical text but rather the lived experience of assembly
democracy, a transformative experience many of us in the Caucus have now
lived through. Knowing that it's not only possible but better for people
to run their own affairs in a democratic manner can invigorate our
movement and raise it to such a point that we will be fully prepared to
take on the hierarchical State and capitalist class which both rule over
us with an iron fist.

\subsubsection{Tenants Unions}\label{tenants-unions}

A dual power tenants union is a group of tenants who work together to
wield collective power against a shared landlord, management company, or
development company in order to improve their housing conditions. While
in general, tenants unions may organize for more affordable, habitable,
and safe housing, the issues that a union decides to organize around are
ultimately decided on by its members. Tenant unions help us develop
working-class power by empowering us to directly apply collective
pressure on our landlords without the permission of city governments or
other third parties. They can also be a force that puts pressure on
local politicians to address abusive landlord behavior in our respective
districts. Because tenant unions are ultimately run by tenants and
community members, they create spaces for ordinary people to grow our
collective power and practice democratic organizing which can be used to
exercise leverage over our exploiters.

As socialists organizing with---and as---tenants, we should work to
bring together such organizations, if they already exist, into city- and
region-wide federations to articulate economic and political demands
that a single tenant union could not do by itself. As with the limited
demands of trade unionism, it isn't simply enough to advocate for lower
rents and more maintenance in our buildings. We must construct a
tenants' movement that cuts across neighborhood boundaries, involves
tenants from different housing situations such as multi-unit buildings,
public housing and SROs, and builds class consciousness through shared
struggle against the common enemy of the owning class. We must fight not
just for better conditions, but for robust community-controlled
collective housing in the form of institutions such as community land
trusts, housing cooperatives, and public housing.

\subsubsection{Mutual Aid}\label{mutual-aid}

\begin{quote}
``The mutual-aid tendency in man has so remote an origin, and is so
deeply interwoven with all the past evolution of the human race, that is
has been maintained by humankind up to the present time, notwithstanding
all vicissitudes of history.''

\href{https://theanarchistlibrary.org/library/petr-kropotkin-mutual-aid-a-factor-of-evolution}{Mutual
Aid: A Factor in Evolution} by Pyotr Kropotkin, 1902
\end{quote}

Providing for the immediate needs of people in our communities can also
act as a support system for our on-the-ground organizing. Mutual aid is
an essential aspect to our biological species and promotes the positive
flourishing of our collective humanity. Our best instincts are those
which serve others, helping each other wherever and however we can,
organizing consciously for what is best for our species and our common
ecology. Mutual aid institutions and networks raise material goods and
services for the entire working class while contributing to other
organizing endeavors. Providing material aid for striking workers,
clothes and food for those facing houselessness, bail bond funds for
marginalized communities who are targeted by the police and the criminal
justice system, material aid for teachers and students who are facing
austerity measures, and disaster relief for those affected by climate
change and natural disasters, are all ways in which we can organize for
a better world, today.

Mutual aid is often painted with the same brush as capitalist ideas of
charity. This ignores the fact that mutual aid is the work we do to
support each other in struggle wherein people take on the responsibility
of caring for each other's needs. As we build skills and share them with
each other, we are able to create a more immediately survivable
environment, challenging alienation and capitalistic relations through
reciprocity and solidarity. This can also be an aid to our struggles.
For example, the wildcat teachers' strikes gained extensive popularity
through providing a replacement for free lunches to low-income students
unable to attend classes due to the strike.

But it's not just that mutual aid can ease material conditions or help
striking workers so that they have more power against the bosses. Its
assault on the existing power structure runs much deeper. Consider this:
if mutual aid can meet the food needs of everyone in a city without them
having to pay for the food, what's the point of paying for food in the
first place? Start asking questions like this, and you can quickly start
to unravel the capitalist economy itself in that local area. Capitalism
is based upon a network of institutions that draw their power from
control and exclusion. Free access is capitalism's poison. By building
up the capacity to universally provide resources on a non-market basis,
we plant the seeds for capitalism's ultimate destruction.

In all this, we must remember that mutual aid runs not from the
socialist movement to the grateful workers but is something workers do
for each other on an egalitarian basis. We must work to ensure our
mutual aid raises each other up as opposed to charity which hands down
from ``on high.''

\subsubsection{Municipalist Syndicalism \& Mass Dual Power
Organizing}\label{municipalist-syndicalism-mass-dual-power-organizing}

Since the resurgence of DSA beginning in 2016 there has been much
discussion about what it means to build a mass movement of working
people to fight for socialism in the United States. We believe that dual
power is critical to building any kind of mass movement with any promise
of creating real socialist democracy. To go further, it is critical to
bring delegates from such organizations together in local councils and
congresses to build solidarity, militancy and knowledge across
communities. Building these networks is key to breaking out of silos and
\textbf{building mass dual power}.

This brings us to \textbf{municipalist syndicalism}---a
\href{https://roarmag.org/essays/municipalist-syndicalism-alex-kolokotronis/}{framework}
that ties industrial democratic organizations, the solidarity economy,
popular assemblies, councils, mutual aid, and the communities in which
they are embedded together to form a confederal cooperative
commonwealth. Such a synthesis is the glue which holds together the dual
power strategy for building libertarian socialism. The cooperative
commonwealth is the linking up of participatory, direct democratic
assemblies with rank-and-file union democracy, with worker-run
cooperatives, and other institutions of dual power as they struggle to
dismantle capitalism and all forms of domination. These organizations,
utilizing direct-democratic processes, mutually reinforce one another
and steadily combine our efforts to intensify our collective power.
Unions can use their initial fundraising advantage to buy spaces the
movement as a whole can use or seed cooperative ventures; cooperatives,
once established, can provide the material resources, such as food and
housing, to help strikes keep going until the bosses are forced to
capitulate; assemblies can organize the fight for legislation that gives
greater support to cooperatives, such as allowing unions to
automatically take over businesses and turn them into coops once they
reach a certain size, as well as help establish mutual aid projects and
community commons which form the initial pilots of programs that can
later be municipalized and established community-wide. All of this
coordination helps to keep internally democratic unions connected to
popular neighborhood assemblies \& councils and the solidarity
economy---and it keeps all of them accountable and democratic. Once
established, dual power institutions that work to expand municipalist
socialism in this or that community then confederate first regionally
and then as a vast international network of similar ``fearless cities''
dedicated to revolution against capitalism and fascism, and to the
dedicated construction of libertarian socialism. This decentralized,
networked, organizational architecture can prioritize the
universalization of economic democracy and the redistribution of goods
and services to all, across respective communities and regions.

\subsection{Conclusion}\label{conclusion}

The \textbf{dual power} strategy is the result of much consideration and
long conversations within the Caucus. Based on our experiences, our
study of past movements, and our participation in current movements
across the country, we believe it is the surest path to anything like
socialism in our lifetimes. Given the ongoing disintegration of the
political and economic order as well as a rapidly shrinking window to
avert the ecological crisis, we must prepare to take meaningful action
sooner rather than later. A dual power approach combines the most
revolutionary aims with the most hard-headed pragmatism, and this is why
we believe it has the best chance of winning.

This strategy should certainly appeal to those already committed to
libertarian socialism. But we think it will also appeal to socialists
who want to believe in the efficacy of apprehending State power to
transform our political economy from capitalism to socialism, but are
made hesitant by an honest reflection upon the profound and tragic
failures that have characterized attempts at state socialism in the
past, from
\href{https://www.theguardian.com/world/2016/nov/15/sweden-relaunches-olof-palme-investigation}{the
assassination of Swedish prime minister Olof Palme} to the
\href{https://www.versobooks.com/blogs/4016-the-coup-in-chile}{CIA-backed
coup in Chile} to the
\href{https://libcom.org/library/soviets-factory-committees-russian-revolution-peter-rachleff}{dissolution
of the factory committees after the Russian revolution}. Seizing state
power intensifies hierarchy and social forms of domination instead of
dissolving them, by making socialist ``fixes'' impositions upon the
broad international working classes rather than a product of their own
self-activity. As we have been reminded by democratic socialists from
Luxemburg, to Harrington, to Öcalan, to Sanders, any socialism worthy of
the name must grow out of the power developed in the community and
working people. For libertarian socialists, that power can only
legitimately be expressed through assemblies and councils, for these
constitute the \textbf{dual power basis} of any transition that is
far-reaching and radical enough to finally replace the functions of
capitalism, the State, and other forms of domination with relationships
rooted in mutual understanding, trust, and care.

Crisis and revolution are in the air. Given the way the world has been
moving, if we engage in the hard work of building the base for dual
power, such a transition may be coming even sooner than many of us
think.

\end{document}